% ----------
% Tech report template
% 
% ----------
\documentclass[11pt, letterpaper, oneside]{article}
\usepackage{geometry}
\usepackage[utf8]{inputenc}
\usepackage[english]{babel}
\usepackage[runin]{abstract}
\usepackage{titling}
\usepackage{booktabs}
\usepackage{fancyhdr}
\usepackage{helvet}
\usepackage{csquotes}
\usepackage{graphicx}
\usepackage{blindtext}
\usepackage{parskip}
\usepackage{etoolbox}
\usepackage{hyperref}
\usepackage{url}
\usepackage{xcolor}


\input{preamble.tex}

% ----------
% Variables
% ----------

\title{\textbf{GECCO 2024---Summary,\\potential future work and exchange opportunities}} % Full title of your tech report
\runningtitle{GECCO'24 Report} % Short title
\author{Xavier F. C. Sánchez Díaz} % Full list of authors
\runningauthor{X. Sánchez Díaz} % Short list of authors
\affiliation{Norwegian University of Science and Technology} % Affiliation e.g. University or Company
\department{Department of Computer Science} % Department or Office
\memoid{GECCO'24 - NAIL} % ID of the tech report
\theyear{2024} % year of the tech report
\mydate{Aug 23, 2024} %the date


% ----------
% actual document
% ----------
\begin{document}
\maketitle

\begin{abstract}
    This is a short compendium of potential areas of interest for my PhD topic, as a result of attending GECCO 2024.
    Nothing seems extraordinarily new, but there are a couple of ideas it is worth to incorporate.
    First, I include a short summary of the whole conference. Then, some of the interesting trends in the community, and then I include a list of authors and advanced topics on multimodal optimisation that could extend our work.
    Finally, a brief description of ROAR-net and its funding opportunities is included.

    \keywords{Evolutionary Computation, Conferences, }
\end{abstract}

\vspace{2.5cm}

{\footnotesize
    \noindent
    Attending the conference was possible thanks to funding from \textbf{The Research Council of Norway} under project number 311284 for the \textbf{Norwegian Open AI Lab}.
}

\thispagestyle{firstpage}

\pagebreak

% ----------
% End of first page
% ----------

\newgeometry{} % Redefine geometries (normal margins)

\section{Summary of the conference}
\label{sec:summary}

My participation in the conference included attending to two different events:

\begin{itemize}
    \item SIGEVO Summer School (S3) 2024
    \item GECCO 2024
\end{itemize}


\subsection{S3 2024}

S3 2024 took place from Wednesday July 10 to Saturday July 13.
It included brief talks about scientific writing, presentation, reviewing and how to do modelling in optimisation problems.

On top of the talks, the summer school had a \textit{coding challenge} (which was extremely challenging indeed---first because they force you to use a specific programming language, and second because they provide a code handout that is old, undocumented and explicitly prohibits the use of industry-ready libraries).
Needless to say, it felt more like a hackathon than a proper challenge, but I assume that is what young students prefer?

Overall, a very useful selection of topics for a first year PhD. I have to say I did not find it very useful myself, but I have been working on problem modelling for around a decade now.

\subsection{GECCO 2024}


GECCO started on Sun Jul 14 with workshops and tutorials for 2 days, and then 2 and a half more days of poster and paper presentations.
This section includes all activities I attended, highlighting those I found the most useful and interesting.
For specific information about a session, please refer to the conference proceedings at \href{https://dl.acm.org/conference/gecco/proceedings}{ACM's website}.

\subsubsection{Tutorials and Workshops}

\begin{enumerate}
    \item \textbf{Tutorial: Lehre's \textit{Runtime analysis of population-based EAs}}
    \item Tutorial: Xue's \textit{Feature Selection and Construction using EAs}
    \item Tutorial: Coello's \textit{Constraint Handling Techniques used with EAs}
    \item Tutorial: Doerr's \textit{A gentle introduction to Theory for Non-Theoreticians}
    \item Special Session: \textit{Women+ @ GECCO}
    \item Workshop: Thomson's \textit{Landscape Aware Heuristic Search}
    \item Workshop: Naujok's \textit{Good Benchmarking Practices for EC}
    \item \textbf{Tutorial: Ochoa's \textit{Landscape Analysis of Optimisation Problems and Algorithms}}
    \item \textbf{Tutorial: Sudholt's \textit{Theory and Practice of Population Diversity in EC}}
    \item Keynote: Toby Walsh's \textit{Generative AI: why all the fuss?}
\end{enumerate}

\subsubsection{Main Conference and Parallel Sessions}

\begin{enumerate}
    \item \textbf{Session: \textit{ROAR-Net}}
    \item Session: \textit{Understanding search trajectories in parameter tuning}
    \item \textbf{Session: \textit{The role of substrate in CA-based EAs}}
    \item Session: \textit{Learning from Offline and Online experiences: A hybrid adaptive operator selection framework}
    \item {Session: \textit{A self adaptive coevolutionary algorithm}}
    \item Session: \textit{Understanding Fitness Landscapes in Morpho-evolution via Local Optima Networks}
    \item Session: \textit{Approximating Pareto Local Optima Solution Networks}
    \item SIGEvo's Impact Session: \textit{Multiple Regression Genetic Programming}
    \item Session: \textit{Learning Traffic Signal Control via GP}
    \item Session: \textit{LLM-based test case generation for GP Agents}
    \item Session: \textit{Enhancing Prediction, Explainability, Inference and Robustness of Decision Trees via Symbolic Regression-Discovered Splits}
    \item Session: \textit{MO Evolutionary Component Effect on Algorithm Behavior}
    \item Session: \textit{A functional analysis approach to symbolic regression}
    \item Session: \textit{LLMs for the automated analysis of optimization algorithms}
    \item Session: \textit{Illustrating the Efficiency of Popular Evolutionary Algorithms using Runtime Analysis}
    \item Session: \textit{Empirical Comparison between MOEAs and Local Search on MO Combinatorial Optimisation Problems}
    \item \textbf{Session: \textit{Markov-chain based Optimization Time Analysis of Bivalent Ant Colony Optimization for Sorting and LeadingOnes}}
    \item \textbf{Session: \textit{Superior Genetic Algorithms for the Target Set Selection Problem Based on Power-Law Parameter Choices and Simple Greedy Heuristics}}
    \item Session: \textit{Genetic Algorithm selection of Interacting Features for selecting biological gene-gene interaction}
    \item \textbf{Session: \textit{The SLO Hierarchy of pseudo-Boolean Functions and Runtime of Evolutionary Algorithms}}
    \item Session: \textit{Evolutionary Preference Sampling for Pareto Set Learning}
    \item Session: \textit{Grahf: a Hyper-heuristic framework for evolving heterogeneous Island Model Topologies}
    \item Session: \textit{Evolving Reliable Differentiating Constraints for the Chance-constrained Maximum Coverage Problem}
    \item Competition: \textit{HUMIES awards: Human-Competitive results}
    \item Keynote: May O'Reilly's \textit{Coevolution in Natural and Artificial Systems}
\end{enumerate}


\section{Discussion: Noticeable Trends}

My overall impression is that there are a few topics in which there were way more submissions than in others.
For example, there was \textbf{only one submission} dealing with \textbf{Hyper-heuristics} while there were multiple of them {solving problems with Genetic Programming}.
A more ``apples to apples'' comparison would be with the \textit{kind} of research done.

For example, I identified two broad types of projects: 1) applications, and 2) theory.
As is usual, there were far more papers on \textbf{applications} (or competitions/benchmarking---actually solving the problem, let's say) than in theory (usually, proposing new problems or proving runtime of specific problems).

Most of the submissions were dealing with \textbf{multi-objective} optimisation rather than testing in single objective domains.
Applications would normally be solved using \textbf{GP}, and most of the optimisation done would be on \textbf{continuous domains}.
Very few of these were actually in \textbf{constrained settings}, and many mention how the hardness of \textbf{synthetic problems} \underline{does not} reflect the behaviour in \textbf{real-life instances}:
synthetic problems might be more tricky to escape from than real life instances, while competition suite problems seem to \underline{not be able} to represent all difficulties that problem instances in real-life would impose.
This occurs for multiple benchmark suites.

Finally, \textbf{Co-evolutionary algorithms} seem to be a \textit{not-so-new} branch of evolutionary computation which looks to be quite promising.
It makes a lot of sense, since it is an easier-to-understand adversarial model (similar to GANs), with the added bonus of the explainability and tracing of the solution which evolutionary computation methods usually provide.

\textbf{Theory and analysis} people, however, have their own agenda and publish multiple things outside these trends.
In theoretical research, the trends are basically laid out by the different research groups (or whomever leads them):
\begin{itemize}
    \item Sudholt's group works on runtime and analysis for \textbf{diversity preservation method}s
    \item Lehre's group works on \textbf{coevolution} and \textbf{runtime analysis of Pseudoboolean problems}
    \item Doerr's group works on \textbf{runtime analysis for specific combinations on parameters and algorithms}, mostly multi-objective
    \item Ochoa's group works on \textbf{landscape analysis}, mostly on \textbf{combinatorial problems} using Local-Optima Networks. Recently they have worked on describing search trajectories (\textbf{algorithm performance}) using Search Trajectory Networks.
\end{itemize}

\section{Possible paths for future work}

In no particular order, the highlighted topics can be formulated as future work with our research for evolutionary computation, landscape analysis and multimodal optimization.

For example, using the \textbf{SLO hierarchy} to continue the work on the \textsc{Triangle} problem we proposed, and doing analysis on it.
This would allow us to formalise the problem and can make it available for other people to use as a test problem.

On the topic of \textbf{Pseudoboolean} functions (PBFs), there is potential on creating a \textit{compendium} of such PBFs.
There are multiple \textit{problem-suites} (like BBOB and COCO) focusing on continuous problems, but I have not seen any website compiling PBFs.
This could be easily set as a result from working on a master student thesis.

\textbf{Strengthening the analysis} we have done before (for example on our GECCO '24 paper, or on our PPSN '24 paper) using \textbf{STNs} is also possible.
In fact, I believe is one of the things we should consider, as Ochoa et al. provided a website to do STNs if provided with a specific output from our algorithms.
This also opens the door to creating more landscape analysis and visualisations for previous papers, e.g., the ambulance problem from our GECCO '23 paper with Magnus Schjølberg and Nicklas Bekkevold.

The \textbf{lack of papers on multimodal optimisation} makes it quite an interesting topic to discuss.
There are hints of its importance, for example for Sudholt's lab where they do \textbf{diversity preservation}, and considering the presence of multiple populations in \textbf{coevolution}.
The fact that we perform landscape analysis means it is important to understand the problem, and multimodal problems are perhaps a bit more interesting considering certain applications.
Maybe \textbf{highlighting these applications} would be wise in order to bring more attention to this matter.

\section{Possible Funding Sources}

An interesting outcome of my networking was talking with \textbf{Per Kristian Lehre} and \textbf{Gabriela Ochoa}.
Both of them are in the \textbf{ROAR Network}, which objective is to create a network of experts in \textit{Randomised Optimisation Algorithms}.
The network provides funding for so-called \textit{short scientific missions} where a scholar visits another institution (hence becoming a \textit{visiting scholar}).
Both Lehre and Ochoa mentioned it was possible to visit their institutions (University of Birmingham and University of Stirling, respectively).

More information about the ROAR-net can be found in \href{https://roar-net.eu/}{their website}.

\section{Conclusions}
\label{sec:conc}

This short report includes a brief description of my activities in the GECCO 2024 conference.
It also highlights the parallel sessions and presentations I found the most interesting, and how they can impact our future research on multimodal optimisation, landscape analysis and in evolutionary computation in general.
Some links are provided to find out more about the funding opportunities from the ROAR-net and its short scientific missions, as well to ACM's proceedings in case more details about a specific presentation are needed.

\section*{Acknowledgements}

Thanks again to funding from \textbf{The Research Council of Norway} under project number 311284 for the \textbf{Norwegian Open AI Lab}, as it allowed me to attend the conference to present our work.

% ----------
% Bibliography
% ----------

% \bibliography{../biblio.bib}
% \bibliographystyle{abbrv}

% \appendix


\end{document}

% ----------
